\documentclass[a4paper, 11pt]{article}
\usepackage{comment} % enables the use of multi-line comments (\ifx \fi)
\usepackage{fullpage} % changes the margin
\usepackage{enumitem}
\usepackage[T1]{fontenc}
\usepackage[polish]{babel}
\usepackage[utf8]{inputenc}
\usepackage{ragged2e}
\usepackage{graphicx}
\usepackage{datatool}
\usepackage{rotating}
\usepackage{placeins}
\usepackage{pdflscape}
\usepackage{float}
\usepackage{listings}
\usepackage{xcolor} % for setting colors

\usepackage{color}

\usepackage{hyperref}
\hypersetup{
colorlinks,
citecolor=black,
filecolor=black,
linkcolor=black,
urlcolor=black
}

\definecolor{dkgreen}{rgb}{0,0.6,0}
\definecolor{gray}{rgb}{0.5,0.5,0.5}
\definecolor{mauve}{rgb}{0.58,0,0.82}


\lstset{frame=tb,
language=SQL,
aboveskip=3mm,
belowskip=3mm,
showstringspaces=false,
columns=flexible,
basicstyle={\small\ttfamily},
numbers=none,
numberstyle=\tiny\color{gray},
keywordstyle=\color[HTML]{EA1212},
commentstyle=\color{dkgreen},
stringstyle=\color{blue},
breaklines=true,
breakatwhitespace=true,
tabsize=3,
otherkeywords={IF, RETURN, IS, BULK, COLLECT, REPLACE, FUNCTION, PROCEDURE, OF, TYPE},
morekeywords={IF, RETURN, IS, BULK, COLLECT, REPLACE, FUNCTION, PROCEDURE, OF, TYPE}
}


\title{Raport z wykonania ćwiczenia Oracle PL/SQL}
\author{Jakub Płotnikowski}
\date{Październik 2019r.}

\begin{document}

    \maketitle
    \tableofcontents


    \newpage

    \section{Tabele - dane podane na początku w skrypcie}
    \begin{flushleft}
        Wykaz tabel w bazie danych.
    \end{flushleft}

    \subsection{Definicje tabel oraz początkowych warunków integralnościowych}
    \lstinputlisting{tables/create_tables.sql}

    \newpage

    \section{Widoki}

    \subsection{Widok WYCIECZKI\_OSOBY}
    \lstinputlisting{views/wycieczki_osoby.sql}

    \subsection{Widok WYCIECZKI\_OSOBY\_POTWIERDZONE}
    \lstinputlisting{views/wycieczki_osoby_potwierdzone.sql}

    \subsection{Widok WYCIECZKI\_OSOBY\_PRZYSZLE}
    \lstinputlisting{views/wycieczki_przyszle.sql}

    \newpage

    \subsection{Widok WYCIECZKI\_MIEJSCA}
    \lstinputlisting{views/wycieczki_miejsca.sql}

    \subsection{Widok DOSTEPNE\_WYCIECZKI}
    \lstinputlisting{views/dostepne_wycieczki_view.sql}

    \subsection{Widok REZERWACJE\_DO\_ANULOWANIA}
    \lstinputlisting{views/rezerwacje_do_anulowania.sql}

    \newpage


    \section{Funkcje pobierające dane}

    \subsection{Definicje typów użytych przy funkcjach}
    \lstinputlisting{procedures/type_definitions.sql}

    \newpage

    \subsection{Funkcja UCZESTNICY\_WYCIECZKI}
    \lstinputlisting{procedures/procedures_extracting_data/uczestnicy_wycieczki.sql}

    \newpage

    \subsection{Funkcja REZERWACJE\_OSOBY}
    \lstinputlisting{procedures/procedures_extracting_data/rezerwacje_osoby.sql}

    \newpage

    \subsection{Funkcja PRZYSZLE\_REZERWACJE\_OSOBY}
    \lstinputlisting{procedures/procedures_extracting_data/przyszle_rezerwacje_osoby.sql}

    \newpage

    \subsection{Funkcja DOSTEPNE\_WYCIECZKI}
    \lstinputlisting{procedures/procedures_extracting_data/dostepne_wycieczki_function.sql}

    \newpage

    \section{Funkcje modyfikujące dane - wersje pierwsze}

    \subsection{Funkcja DODAJ\_REZERWACJE}
    \lstinputlisting{procedures/procedures_modifying_data/dodaj_rezerwacje.sql}

    \newpage

    \subsection{Funkcja ZMIEN\_STATUS\_REZERWACJI}
    \lstinputlisting{procedures/procedures_modifying_data/zmien_status_rezerwacji.sql}

    \newpage

    \subsection{Funkcja ZMIEN\_LICZBE\_MIEJSC}
    \lstinputlisting{procedures/procedures_modifying_data/zmien_liczbe_miejsc.sql}

    \newpage


    \section{Definicja tabeli dziennikującej}

    \subsection{Tabela REZERWACJE\_LOG}
    \lstinputlisting{tables/create_rezerwacje_log.sql}

    \newpage


    \section{Widoki z użyciem dodanej kolumny \newline LICZBA\_WOLNYCH\_MIEJSC
    w tabeli WYCIECZKI}

    \subsection{Widok DOSTEPNE\_WYCIECZKI2}
    \lstinputlisting{views/dostepne_wycieczki_view2.sql}


    \subsection{Widok WYCIECZKI\_MIEJSCA2}
    \lstinputlisting{views/wycieczki_miejsca2.sql}


    \section{Procedura przelicz}
    \lstinputlisting{procedures/przelicz.sql}

    \newpage

    \section{Funkcje pobierające dane z użyciem dodanej kolumny
    \newline LICZBA\_WOLNYCH\_MIEJSC}

    \subsection{Funkcja DOSTEPNE\_WYCIECZKI2}
    \lstinputlisting{procedures/procedures_extracting_data/dostepne_wycieczki_function2.sql}

    \newpage


    \section{Funkcje modyfikujące dane z użyciem tabeli dziennikującej oraz z dodaną kolumną LICZBA\_WOLNYCH\_MIEJSC w tabeli WYCIECZKI}

    \subsection{Wyszukanie sekwencji dla poszczególnych obiektów}
    \lstinputlisting{others/sequences.sql}

    Sekwencja dla PK tabeli REZERWACJE jest u mnie w bazie nastepująca:

    \begin{lstlisting}[language={}, caption={}]
        "ISEQ$$_79074"
    \end{lstlisting}


    \subsection{Funkcja DODAJ\_REZERWACJE2}
    \lstinputlisting{procedures/procedures_modifying_data/dodaj_rezerwacje2.sql}

    \newpage

    \subsection{Funkcja ZMIEN\_STATUS\_REZERWACJI2}
    \lstinputlisting{procedures/procedures_modifying_data/zmien_status_rezerwacji2.sql}

    \newpage

    \subsection{Funkcja ZMIEN\_LICZBE\_MIEJSC2}
    \lstinputlisting{procedures/procedures_modifying_data/zmien_liczbe_miejsc2.sql}

    \newpage

    \section{Triggery}

    \subsection{Trigger obsługujący dodanie rezerwacji - DODAJ\_NOWA\_REZERWACJE}
    \lstinputlisting{triggers/dodaj_nowa_rezerwacje_trigger.sql}

    \subsection{Trigger obsługujący zmiane statusu - ZMIEN\_STATUS}
    \lstinputlisting{triggers/zmien_status_trigger.sql}

    \newpage

    \subsection{Trigger zabraniający usunięcia rezerwacji - USUN\_REZERWACJE}
    \lstinputlisting{triggers/usun_rezerwacje_trigger.sql}

    \subsection{Trigger obsługujący zmianę liczby miejsc
    \newline na poziomie wycieczki - ZMIEN\_LICZBE\_MIEJSC}
    \lstinputlisting{triggers/zmien_liczbe_miejsc_trigger.sql}


    \section{Uaktualnione procedury modyfikujące dane po dodaniu triggerów}

    \subsection{Funkcja DODAJ\_REZERWACJE3}
    \lstinputlisting{procedures/procedures_modifying_data/dodaj_rezerwacje3.sql}

    \newpage

    \subsection{Funkcja ZMIEN\_STATUS\_REZERWACJI3}
    \lstinputlisting{procedures/procedures_modifying_data/zmien_status_rezerwacji3.sql}

    \newpage

    \subsection{Funkcja ZMIEN\_LICZBE\_MIEJSC3}
    \lstinputlisting{procedures/procedures_modifying_data/zmien_liczbe_miejsc3.sql}

\end{document}
