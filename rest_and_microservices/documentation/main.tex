\documentclass[a4paper, 11pt]{article}
\usepackage{comment} % enables the use of multi-line comments (\ifx \fi) 
\usepackage{fullpage} % changes the margin
\usepackage{enumitem}
\usepackage[T1]{fontenc}
\usepackage[polish]{babel}
\usepackage[utf8]{inputenc}
\usepackage{ragged2e}
\usepackage{graphicx}
\usepackage{datatool}
\usepackage{rotating}
\usepackage{placeins}
\usepackage{pdflscape}
\usepackage{float}
\usepackage{listings}
\usepackage{xcolor} % for setting colors

\usepackage{color}

\usepackage{hyperref}
\hypersetup{
colorlinks,
citecolor=black,
filecolor=black,
linkcolor=black,
urlcolor=black
}

\definecolor{lightgray}{rgb}{.9,.9,.9}
\definecolor{darkgray}{rgb}{.4,.4,.4}
\definecolor{purple}{rgb}{0.65, 0.12, 0.82}

\lstdefinelanguage{JavaScript}{
  keywords={typeof, new, true, false, catch, function, return, null, catch, switch, var, if, in, while, do, else, case, break},
  keywordstyle=\color{blue}\bfseries,
  ndkeywords={class, export, boolean, throw, implements, import, this},
  ndkeywordstyle=\color{darkgray}\bfseries,
  identifierstyle=\color{black},
  sensitive=false,
  comment=[l]{//},
  morecomment=[s]{/*}{*/},
  commentstyle=\color{purple}\ttfamily,
  stringstyle=\color{red}\ttfamily,
  morestring=[b]',
  morestring=[b]"
}

\lstset{
   language=JavaScript,
   backgroundcolor=\color{white},
   extendedchars=true,
   basicstyle=\footnotesize\ttfamily,
   showstringspaces=false,
   showspaces=false,
   numbers=left,
   numberstyle=\footnotesize,
   numbersep=9pt,
   tabsize=2,
   breaklines=true,
   showtabs=false,
   captionpos=b
}


\definecolor{dkgreen}{rgb}{0,0.6,0}
\definecolor{gray}{rgb}{0.5,0.5,0.5}
\definecolor{mauve}{rgb}{0.58,0,0.82}

\title{Raport z wykonania ćwiczenia REST, Mikroserwisy}
\author{Jakub Płotnikowski}
\date{Styczeń 2020}

\begin{document}

\maketitle
\tableofcontents

 \newpage 

\section{Kod z zadań 1, 2, 3}

Jako, że zadania 1, 2 oraz 3 wykonałem oraz oddałem na zajęciach, wklejam tylko ich kod bez dodatkowego omówienia.

 \subsection{Zadanie 1}
 \lstinputlisting{source_code/01/app.js}
 
  \newpage 

 \subsection{Zadanie 2}
 \lstinputlisting{source_code/02/app.js}
 
  \newpage 

 \subsection{Zadanie 3}
 \lstinputlisting{source_code/03/app.js}
 
 \newpage 
 
 \section{Zadanie 4}
 
 \subsection{Treść zadania}
 Wyprowadzić i przeanalizować wynik metody GET dla bazowego URLa.
Zaobserwować rezultaty dla URLa zawierającego numer (id) pacjenta w zależności od użytej metody HTTP.
   Dla testowania wygodnie jest użyć wywołania curl -X
   Przeanalizować różnice w logice poszczególnych implementacji kodu dla obsługi tych metod.
Uwaga: Zwrócić uwagę na fragment kodu pomiędzy db i średnikiem, który odwołuje się do kodu pakietu lowdb, zainicjowanego na początku.
Zamieścić w raporcie przykładowe wyniki (i komentarz do nich).
 
 \subsection{Pobranie listy pacjentów}
 \lstinputlisting{source_code/04/get_patients_empty.js}
 
 \subsection{Dodanie przykładowego pacjenta}
 \lstinputlisting{source_code/04/add_item.js}
 
 \subsection{Zmiana imienia i nazwiska pacjenta}
 \lstinputlisting{source_code/04/change_item.js}

 \subsection{Ponowne pobranie listy pacjentów}
 \lstinputlisting{source_code/04/get_patients_empty.js}

 \subsection{Usunięcie pacjenta o id równym 1}
 \lstinputlisting{source_code/04/delete_patient.js}

 \subsection{Ponowne pobranie listy pacjentów}
 \lstinputlisting{source_code/04/get_patients_empty.js}
\end{document}
